% \iffalse meta-comment
%
%% File: xelatexja.dtx
% 
% Copyright (c) 2021 Yukimasa Morimi (h20y6m)
%   GitHub:   https://github.com/h20y6m
%
% This package is distributed under the MIT License.
%
% -----------------------------------------------------------------------
%
%<*driver>
\documentclass[full]{l3doc}
\usepackage{xelatexja}
\usepackage[japanese]{babel}
\usepackage{bxtexlogo}
\bxtexlogoimport{*}
\usepackage{indentfirst}
\setlength{\parindent}{10pt}
\setlength{\textwidth}{\dimexpr10pt*(\numexpr\textwidth/(10*65536)\relax)\relax}
\renewcommand{\labelitemi}{•}
\ExplSyntaxOn
\xeja_set_prebreakpenalty:nn { \c_xeja_kanji_smallkana_ic } { 100 }
\ExplSyntaxOff
\begin{document}
  \DocInput{\jobname.dtx}
\end{document}
%</driver>
%
%<*package>
\NeedsTeXFormat{LaTeX2e}[2020-10-01]
\RequirePackage{expl3}[2020-12-07]
\ProvidesExplPackage{xelatexja}{2021/01/02}{0.0.1}
  {Japanese typesetting with XeLaTeX}
%</package>
%
% \fi
%
% \title{\textsf{\XeLaTeX-ja}パッケージ}
% \author{森見幸正 (h20y6m)}
% \date{2021年1月2日}
%
% \maketitle
%
% \begin{abstract}
%   これは\XeLaTeX で和文組版を行う実験的なパッケージである。
% \end{abstract}
%
% \tableofcontents
%
% \begin{documentation}
%
% \section{動作要件}
%
% 本パッケージは\XeLaTeX 上で動作する。
%
% \end{documentation}
%
% \StopEventually{}
%
% \begin{implementation}
%
% \section{\pkg{xelatexja}の実装}
%
%    \begin{macrocode}
%<*package>
%    \end{macrocode}
%
%    \begin{macrocode}
%<@@=xeja>
%    \end{macrocode}
%
% \XeTeX が必要。
%    \begin{macrocode}
\msg_new:nnn { xelatexja } { needs-xetex }
  { XeLaTeX-ja~needs~XeTeX. }
\sys_if_engine_xetex:F
  {
    \msg_critical:nn { xelatexja } { needs-xetex }
  }
%    \end{macrocode}
%
% 依存パッケージの読込。
%    \begin{macrocode}
\RequirePackage{l3keys2e,xparse}
%    \end{macrocode}
%
% \subsection{変数}
%
% \begin{macro}{\g_xeja_tate_document_bool}
% 文書全体が縦組かどうかを表す変数。
%    \begin{macrocode}
\bool_new:N \g_xeja_tate_document_bool
%    \end{macrocode}
% \end{macro}
% \begin{macro}{\l_xeja_tate_text_bool}
% 現在の組方向が縦組かどうかを表す変数。
%    \begin{macrocode}
\bool_new:N \l_xeja_tate_text_bool
%    \end{macrocode}
% \end{macro}
%
%    \begin{macrocode}
\tl_new:N \l_xeja_kanji_tl
\tl_set:Nn \l_xeja_kanji_tl { 0\zw plus 0.25\zw minus 0\zw }
%    \end{macrocode}
%
%    \begin{macrocode}
\tl_new:N \l_xeja_xkanji_tl
\tl_set:Nn \l_xeja_xkanji_tl { 0.25\zw plus 0.25\zw minus 0.125\zw }
%    \end{macrocode}
%
% \begin{macro}{\zw}
% 全角寸法。
%    \begin{macrocode}
\dim_new:N \zw
%    \end{macrocode}
% |\set@fontsize|で|\zw|を設定する。
%    \begin{macrocode}
\cs_new_eq:NN \@@_original_set_fontsize:nnn \set@fontsize
\cs_set:Npn \set@fontsize #1#2#3
  {
    \@@_original_set_fontsize:nnn {#1} {#2} {#3}
    \dim_set:Nn \zw { \f@size pt }
  }
%    \end{macrocode}
% \end{macro}
%
% \begin{macro}{\l_@@_tmpa_tl}
% 一時変数。
%    \begin{macrocode}
\tl_new:N \l_@@_tmpa_tl
%    \end{macrocode}
% \end{macro}
%
% \subsection{オプション}
%
%    \begin{macrocode}
\keys_define:nn { xelatexja }
  {
    tate .bool_gset:N = \g_xeja_tate_document_bool,
  }
\ProcessKeysOptions { xelatexja }
%    \end{macrocode}
%
% \subsection{フォント}
%
% 和文フォントエンコーディング。^^A
% 横組みはJY4、縦組みはJT4。
%    \begin{macrocode}
\str_const:Nn \c_xeja_yoko_encoding_str { JY4 }
\str_const:Nn \c_xeja_tate_encoding_str { JT4 }
%    \end{macrocode}
%
%    \begin{macrocode}
\prop_new:N \g_@@_kanji_family_prop
\prop_new:N \g_@@_kanji_shape_prop
%    \end{macrocode}
%
%    \begin{macrocode}
\tl_new:N \l_@@_kanji_family_tl
%    \end{macrocode}
%
%    \begin{macrocode}
\cs_new:Npn \xeja_declare_kanji_family:nn #1#2
  {
    \prop_gput:Nnn \g_@@_kanji_family_prop {#1} {#2}
  }
\cs_generate_variant:Nn \xeja_declare_kanji_family:nn { xn }
%    \end{macrocode}
%
%    \begin{macrocode}
\cs_new:Npn \xeja_declare_kanji_shape:nnnn #1#2#3#4
  {
    \prop_gput:Nnn \g_@@_kanji_shape_prop { #1 / #2 / #3 } {#4}
  }
\cs_generate_variant:Nn \xeja_declare_kanji_shape:nnnn { xxxx }
%    \end{macrocode}
%
%    \begin{macrocode}
\cs_new:Npn \xeja_kanji_family:n #1
  {
    \tl_set:Nx \l_@@_kanji_family_tl {#1}
  }
%    \end{macrocode}
%
%    \begin{macrocode}
\cs_new:Npn \@@_new_font:Nnn #1#2#3
  {
    \tex_global:D \tex_font:D #1 = "#2" ~ at ~ #3 pt ~
  }
\cs_generate_variant:Nn \@@_new_font:Nnn { Nxn, cxn }
%    \end{macrocode}
%
%    \begin{macrocode}
\tl_new:N \l_@@_kanji_default_shape_tl
\tl_new:N \l_@@_kanji_font_shape_tl
\tl_new:N \l_@@_kanji_font_name_tl
%    \end{macrocode}
%
%    \begin{macrocode}
\cs_new:Npn \@@_select_kanji_font_aux:nn #1#2
  {
    \tl_set:Nx \l_@@_kanji_default_shape_tl
      { \l_@@_kanji_family_tl / m / n }
    \tl_set:Nx \l_@@_kanji_font_shape_tl
      { \l_@@_kanji_family_tl / \f@series / \f@shape }
    \tl_set:Nx \l_@@_kanji_font_name_tl
      { #1 / \l_@@_kanji_font_shape_tl / \f@size }
    \cs_if_exist:cF
      { \l_@@_kanji_font_name_tl }
      {
        \exp_args:Ncx \@@_select_kanji_font_aux_ii:NnnF
          { \l_@@_kanji_font_name_tl }
          { \l_@@_kanji_font_shape_tl }
          {#2}
          {
            \exp_args:Ncx \@@_select_kanji_font_aux_ii:NnnTF
              { \l_@@_kanji_font_name_tl }
              { \l_@@_kanji_default_shape_tl }
              {#2}
              {
                \msg_warning:nnxx { xelatexja } { font / instead }
                  { \l_@@_kanji_font_shape_tl }
                  { \l_@@_kanji_default_shape_tl }
              }
              {
                \msg_warning:nnx { xelatexja } { font / undefined }
                  { \l_@@_kanji_font_shape_tl }
                \cs_gset_eq:cN { \l_@@_kanji_font_name_tl } \prg_do_nothing:
              }
          }
      }
    \use:c { \l_@@_kanji_font_name_tl }
  }
\prg_new_conditional:Npnn \@@_select_kanji_font_aux_ii:Nnn #1#2#3
  { T, F, TF }
  {
    \prop_get:NnNTF \g_@@_kanji_shape_prop {#2}
      \l_@@_tmpa_tl
      {
        \tl_if_empty:nF {#3}
          {
            \tl_if_in:NnTF \l_@@_tmpa_tl { : }
              { \tl_put_right:Nn \l_@@_tmpa_tl { , #3 } }
              { \tl_put_right:Nn \l_@@_tmpa_tl { : #3 } }
          }
        \@@_new_font:Nxn #1
          { \l_@@_tmpa_tl }
          { \f@size }
        \prg_return_true:
      }
      {
        \prg_return_false:
      }
  }
\msg_new:nnn { xelatexja } { font / instead }
  { Kanji~shape~`#1'~undefined.~using `#2'~instead. }
\msg_new:nnn { xelatexja } { font / undefined }
  { Kanji~shape~`#1'~undefined. }
%    \end{macrocode}
%
%    \begin{macrocode}
\cs_new:Npn \xeja_select_yoko_kanji_font:
  {
    \@@_select_kanji_font_aux:nn
      { \c_xeja_yoko_encoding_str }
      {}
  }
\cs_new:Npn \xeja_select_tate_kanji_font:
  {
    \@@_select_kanji_font_aux:nn
      { \c_xeja_tate_encoding_str }
      { vertical }
  }
%    \end{macrocode}
%
% \subsubsection{フォント設定}
%
%    \begin{macrocode}
\xeja_declare_kanji_family:nn { mc } {}
\xeja_declare_kanji_family:nn { gt } {}
%    \end{macrocode}
%
%    \begin{macrocode}
\xeja_declare_kanji_shape:nnnn { mc } { m } { n }
  { [HaranoAjiMincho-Regular.otf]:+fwid }
\xeja_declare_kanji_shape:nnnn { gt } { m } { n }
  { [HaranoAjiGothic-Medium.otf]:+fwid }
\xeja_declare_kanji_shape:nnnn { mc } { b } { n }
  { [HaranoAjiGothic-Medium.otf]:+fwid }
\xeja_declare_kanji_shape:nnnn { gt } { b } { n }
  { [HaranoAjiGothic-Medium.otf]:+fwid }
\xeja_declare_kanji_shape:nnnn { mc } { bx } { n }
  { [HaranoAjiGothic-Medium.otf]:+fwid }
\xeja_declare_kanji_shape:nnnn { gt } { bx } { n }
  { [HaranoAjiGothic-Medium.otf]:+fwid }
%    \end{macrocode}
%
%    \begin{macrocode}
\xeja_kanji_family:n { mc }
%    \end{macrocode}
%
%    \begin{macrocode}

\cs_new:Npn \xeja_enter_kanji_font:
  {
    \bool_if:NTF \l_xeja_tate_text_bool
      { \xeja_select_tate_kanji_font: }
      { \xeja_select_yoko_kanji_font: }
  }
\cs_new:Npn \xeja_leave_kanji_font:
  {
    % \iow_term:x { \f@encoding / \f@family / \f@series / \f@shape / \f@size }
    \selectfont
  }
%    \end{macrocode}
%
% \subsection{文字間トークン挿入機能}
%
% 文字間トークン挿入機能の有効化
%    \begin{macrocode}
\tex_XeTeXinterchartokenstate:D = 1 ~
%    \end{macrocode}
% 
% \subsubsection{文字クラス定義}
%
% \begin{itemize}
% \item 欧文文字
%    \begin{macrocode}
\int_const:Nn \c_xeja_alpha_ic { 0 }
%    \end{macrocode}
% \item 和文文字
%    \begin{macrocode}
\newXeTeXintercharclass \c_xeja_kanji_ic
%    \end{macrocode}
% \item 始め括弧類(cl-01)
%    \begin{macrocode}
\newXeTeXintercharclass \c_xeja_kanji_openingbrackets_ic
%    \end{macrocode}
% \item 終わり括弧類(cl-02)
%    \begin{macrocode}
\newXeTeXintercharclass \c_xeja_kanji_closingbrackets_ic
%    \end{macrocode}
% \item ハイフン類(cl-03)
%    \begin{macrocode}
\newXeTeXintercharclass \c_xeja_kanji_hyphens_ic
%    \end{macrocode}
% \item 区切り約物(cl-04)
%    \begin{macrocode}
\newXeTeXintercharclass \c_xeja_kanji_dividingpunctuationmarks_ic
%    \end{macrocode}
% \item 中点類(cl-05)
%    \begin{macrocode}
\newXeTeXintercharclass \c_xeja_kanji_middledots_ic
%    \end{macrocode}
% \item 句点類(cl-06)
%    \begin{macrocode}
\newXeTeXintercharclass \c_xeja_kanji_fullstops_ic
%    \end{macrocode}
% \item 読点類(cl-07)
%    \begin{macrocode}
\newXeTeXintercharclass \c_xeja_kanji_commas_ic
%    \end{macrocode}
% \item 分離禁止文字(cl-08)
%    \begin{macrocode}
\newXeTeXintercharclass \c_xeja_kanji_inseparablecharacters_ic
%    \end{macrocode}
% \item 繰返し記号(cl-09)
%    \begin{macrocode}
\newXeTeXintercharclass \c_xeja_kanji_iterationmarks_ic
%    \end{macrocode}
% \item 長音記号(cl-10)
%    \begin{macrocode}
\newXeTeXintercharclass \c_xeja_kanji_prolongedsoundmark_ic
%    \end{macrocode}
% \item 小書き仮名(cl-11)
%    \begin{macrocode}
\newXeTeXintercharclass \c_xeja_kanji_smallkana_ic
%    \end{macrocode}
% \item 前置省略記号(cl-12)
%    \begin{macrocode}
\newXeTeXintercharclass \c_xeja_kanji_prefixedabbreviations_ic
%    \end{macrocode}
% \item 平仮名(cl-15)
%    \begin{macrocode}
\newXeTeXintercharclass \c_xeja_kanji_hiragana_ic
%    \end{macrocode}
% \item 片仮名(cl-16)
%    \begin{macrocode}
\newXeTeXintercharclass \c_xeja_kanji_katakana_ic
%    \end{macrocode}
% \item 境界
%    \begin{macrocode}
\int_const:Nn \c_xeja_bound_ic { 4095 }
%    \end{macrocode}
% \end{itemize}
%
% \begin{macro}{\g_@@_kanji_class_seq}
% 和文文字クラス。
%    \begin{macrocode}
\seq_new:N \g_@@_kanji_class_seq
\seq_gset_from_clist:Nn \g_@@_kanji_class_seq
  {
    \c_xeja_kanji_ic ,
    \c_xeja_kanji_openingbrackets_ic ,
    \c_xeja_kanji_closingbrackets_ic ,
    \c_xeja_kanji_hyphens_ic ,
    \c_xeja_kanji_dividingpunctuationmarks_ic ,
    \c_xeja_kanji_middledots_ic ,
    \c_xeja_kanji_fullstops_ic ,
    \c_xeja_kanji_commas_ic ,
    \c_xeja_kanji_inseparablecharacters_ic ,
    \c_xeja_kanji_iterationmarks_ic ,
    \c_xeja_kanji_prolongedsoundmark_ic ,
    \c_xeja_kanji_smallkana_ic ,
    \c_xeja_kanji_prefixedabbreviations_ic ,
    \c_xeja_kanji_hiragana_ic ,
    \c_xeja_kanji_katakana_ic
  }
%    \end{macrocode}
% \end{macro}
%
% \subsubsection{文字間トークン}
%
%    \begin{macrocode}
\cs_new:Npn \@@_clear_interchar_tokens:nn #1#2
  {
    \@@_set_interchar_tokens:nnn {#1} {#2} {}
  }
%    \end{macrocode}
%
%    \begin{macrocode}
\cs_new:Npn \@@_set_interchar_tokens:nnn #1#2#3
  {
    \tex_XeTeXinterchartoks:D \int_eval:n {#1} ~ \int_eval:n {#2} = {#3}
  }
\cs_generate_variant:Nn \@@_set_interchar_tokens:nnn
  { nnV, nnv, nno, nnf, nnx }
%    \end{macrocode}
%
%    \begin{macrocode}
\cs_new:Npn \@@_put_left_interchar_tokens:nnn #1#2#3
  {
    \@@_set_interchar_tokens:nnx {#1} {#2}
      {
        \exp_not:n {#3}
        \@@_the_interchar_tokens:nn {#1} {#2}
      }
  }
\cs_generate_variant:Nn \@@_put_left_interchar_tokens:nnn { nnV, nno, nnx }
%    \end{macrocode}
%
%    \begin{macrocode}
\cs_new:Npn \@@_put_right_interchar_tokens:nnn #1#2#3
  {
    \@@_set_interchar_tokens:nnx {#1} {#2}
      {
        \@@_the_interchar_tokens:nn {#1} {#2}
        \exp_not:n {#3}
      }
  }
\cs_generate_variant:Nn \@@_put_right_interchar_tokens:nnn
  { nnV, nno, nnx }
%    \end{macrocode}
%
%    \begin{macrocode}
\cs_new:Npn \@@_the_interchar_tokens:nn #1#2
  {
    \tex_the:D \tex_XeTeXinterchartoks:D \int_eval:n {#1} ~ \int_eval:n {#2} ~
  }
%    \end{macrocode}
%
% 行末禁則
%    \begin{macrocode}
\prop_new:N \l_@@_postbreakpenalty_prop
\cs_new:Npn \xeja_set_postbreakpenalty:nn #1#2
  {
    \prop_put:Nxx \l_@@_postbreakpenalty_prop
      { \int_eval:n {#1} }
      { \int_eval:n {#2} }
  }
\clist_map_inline:nn
  { 
    \c_xeja_kanji_openingbrackets_ic ,
    \c_xeja_kanji_prefixedabbreviations_ic ,
  }
  {
    \xeja_set_postbreakpenalty:nn {#1} { 10000 }
  }
%    \end{macrocode}
%
% 行頭禁則
%    \begin{macrocode}
\prop_new:N \l_@@_prebreakpenalty_prop
\cs_new:Npn \xeja_set_prebreakpenalty:nn #1#2
  {
    \prop_put:Nxx \l_@@_prebreakpenalty_prop
      { \int_eval:n {#1} }
      { \int_eval:n {#2} }
  }
\clist_map_inline:nn
  { 
    \c_xeja_kanji_closingbrackets_ic ,
    \c_xeja_kanji_hyphens_ic ,
    \c_xeja_kanji_dividingpunctuationmarks_ic ,
    \c_xeja_kanji_middledots_ic ,
    \c_xeja_kanji_fullstops_ic ,
    \c_xeja_kanji_commas_ic ,
    \c_xeja_kanji_iterationmarks_ic ,
    \c_xeja_kanji_prolongedsoundmark_ic ,
    \c_xeja_kanji_smallkana_ic
  }
  {
    \xeja_set_prebreakpenalty:nn {#1} { 10000 }
  }
%    \end{macrocode}
%
% 文字間のアキを定義する。
%    \begin{macrocode}
\prop_new:N \l_@@_interchar_skip_prop
\cs_new:Npn \xeja_set_interchar_skip:nnn #1#2#3
  { 
    \exp_args:NNx \prop_put:Nnn \l_@@_interchar_skip_prop
      { \int_eval:n {#1} / \int_eval:n {#2} }
      {#3}
  }
\seq_map_inline:Nn \g_@@_kanji_class_seq
  {
    \xeja_set_interchar_skip:nnn {#1} { \c_xeja_kanji_openingbrackets_ic }
      { 0.5\zw plus 0.25\zw minus 0.25\zw }
    \xeja_set_interchar_skip:nnn { \c_xeja_kanji_closingbrackets_ic } {#1}
      { 0.5\zw plus 0.25\zw minus 0.25\zw }
    \xeja_set_interchar_skip:nnn { \c_xeja_kanji_fullstops_ic } {#1}
      { 0.5\zw plus 0.25\zw minus 0.25\zw }
    \xeja_set_interchar_skip:nnn { \c_xeja_kanji_commas_ic } {#1}
      { 0.5\zw plus 0.25\zw minus 0.25\zw }
  }
\seq_map_inline:Nn \g_@@_kanji_class_seq
  {
    \xeja_set_interchar_skip:nnn { \c_xeja_kanji_openingbrackets_ic } {#1}
      { 0\zw }
    \xeja_set_interchar_skip:nnn {#1} { \c_xeja_kanji_closingbrackets_ic }
      { 0\zw }
    \xeja_set_interchar_skip:nnn {#1} { \c_xeja_kanji_fullstops_ic }
      { 0\zw }
    \xeja_set_interchar_skip:nnn {#1} { \c_xeja_kanji_commas_ic }
      { 0\zw }
  }
\seq_map_inline:Nn \g_@@_kanji_class_seq
  {
    \xeja_set_interchar_skip:nnn {#1} { \c_xeja_kanji_middledots_ic }
      { 0.25\zw plus 0.25\zw minus 0.25\zw }
    \xeja_set_interchar_skip:nnn { \c_xeja_kanji_middledots_ic } {#1}
      { 0.25\zw plus 0.25\zw minus 0.25\zw }
  }
\xeja_set_interchar_skip:nnn
  { \c_xeja_kanji_middledots_ic } { \c_xeja_kanji_middledots_ic }
  { 0.5\zw plus 0.25\zw minus 0.25\zw }
\xeja_set_interchar_skip:nnn
  { \c_xeja_kanji_inseparablecharacters_ic }
  { \c_xeja_kanji_inseparablecharacters_ic }
  { 0\zw }
%    \end{macrocode}
%
% \subsubsection{和文-和文間}
%
% 禁則ペナルティー。
%    \begin{macrocode}
\cs_set:Npn \@@_tmp:nn #1#2
  {
    \@@_put_left_interchar_tokens:nnn {#1} {#2}
      {
        \prop_get:NnNT \l_@@_postbreakpenalty_prop {#1} \l_@@_tmpa_tl
          {
            \tex_penalty:D \l_@@_tmpa_tl
          }
        \prop_get:NnNT \l_@@_prebreakpenalty_prop {#2} \l_@@_tmpa_tl
          {
            \tex_penalty:D \l_@@_tmpa_tl
          }
      }
  }
\seq_map_inline:Nn \g_@@_kanji_class_seq
  {
    \seq_map_inline:Nn \g_@@_kanji_class_seq
      {
        \exp_args:Nff \@@_tmp:nn { \int_eval:n {#1} } { \int_eval:n {##1} }
      }
  }
%    \end{macrocode}
%
% アキ調整。
%    \begin{macrocode}
\cs_set:Npn \@@_tmp:nn #1#2
  {
    \@@_put_right_interchar_tokens:nnn {#1} {#2}
      {
        \prop_get:NnNTF \l_@@_interchar_skip_prop { #1 / #2 }
          \l_@@_tmpa_tl
          {
            \skip_horizontal:n { \l_@@_tmpa_tl }
          }
          {
            \skip_horizontal:n { \l_xeja_kanji_tl }
          }
      }
  }
\seq_map_inline:Nn \g_@@_kanji_class_seq
  {
    \seq_map_inline:Nn \g_@@_kanji_class_seq
      {
        \exp_args:Nff \@@_tmp:nn { \int_eval:n {#1} } { \int_eval:n {##1} }
      }
  }
%    \end{macrocode}
%
% 次の文字の字幅調整。
%    \begin{macrocode}
\seq_map_inline:Nn \g_@@_kanji_class_seq
  {
    \@@_put_right_interchar_tokens:nnn
      {#1} { \c_xeja_kanji_openingbrackets_ic }
      { \tex_kern:D -0.5\zw }
    \@@_put_right_interchar_tokens:nnn
      {#1} { \c_xeja_kanji_middledots_ic }
      { \tex_kern:D -0.25\zw }
  }
%    \end{macrocode}
%
% 前の文字の字幅調整。
%    \begin{macrocode}
\seq_map_inline:Nn \g_@@_kanji_class_seq
  {
    \@@_put_left_interchar_tokens:nnn
      { \c_xeja_kanji_closingbrackets_ic } {#1}
      { \tex_kern:D -0.5\zw }
    \@@_put_left_interchar_tokens:nnn
      { \c_xeja_kanji_middledots_ic } {#1}
      { \tex_kern:D -0.25\zw }
    \@@_put_left_interchar_tokens:nnn
      { \c_xeja_kanji_fullstops_ic } {#1}
      { \tex_kern:D -0.5\zw }
    \@@_put_left_interchar_tokens:nnn
      { \c_xeja_kanji_commas_ic } {#1}
      { \tex_kern:D -0.5\zw }
  }
%    \end{macrocode}
%
% \subsubsection{欧文-和文間}
%
% 禁則ペナルティー。
%    \begin{macrocode}
\cs_set:Npn \@@_tmp:n #1
  {
    \@@_put_left_interchar_tokens:nnn { 0 } {#1}
      {
        \prop_get:NnNT \l_@@_prebreakpenalty_prop {#1} \l_@@_tmpa_tl
          {
            \tex_penalty:D \l_@@_tmpa_tl
          }
      }
  }
\seq_map_inline:Nn \g_@@_kanji_class_seq
  {
    \exp_args:Nff \@@_tmp:nn { 0 } {#1}
  }
%    \end{macrocode}
%
% 和欧文間空白。
%    \begin{macrocode}
\clist_map_inline:nn
  {
    \c_xeja_kanji_ic ,
    \c_xeja_kanji_openingbrackets_ic ,
    \c_xeja_kanji_hyphens_ic ,
    \c_xeja_kanji_dividingpunctuationmarks_ic ,
    \c_xeja_kanji_middledots_ic ,
    \c_xeja_kanji_inseparablecharacters_ic ,
    \c_xeja_kanji_iterationmarks_ic ,
    \c_xeja_kanji_prolongedsoundmark_ic ,
    \c_xeja_kanji_smallkana_ic ,
    \c_xeja_kanji_prefixedabbreviations_ic ,
    \c_xeja_kanji_hiragana_ic ,
    \c_xeja_kanji_katakana_ic ,
  }
  {
    \@@_put_right_interchar_tokens:nnn { 0 } {#1}
      {
        \skip_horizontal:n { \l_xeja_xkanji_tl }
      }
  }
%    \end{macrocode}
%
% 次の文字の字幅調整。
%    \begin{macrocode}
\@@_put_right_interchar_tokens:nnn
  { 0 } { \c_xeja_kanji_openingbrackets_ic }
  { \tex_kern:D -0.5\zw }
\@@_put_right_interchar_tokens:nnn
  { 0 } { \c_xeja_kanji_middledots_ic }
  { \tex_kern:D -0.25\zw }
%    \end{macrocode}
%
% フォント切り替え。
%    \begin{macrocode}
\seq_map_inline:Nn \g_@@_kanji_class_seq
  {
    \@@_put_left_interchar_tokens:nnn { 0 } {#1}
      {
        \xeja_enter_kanji_font:
      }
  }
%    \end{macrocode}
%
% \subsubsection{和文-欧文間}
%
% 禁則ペナルティー。
%    \begin{macrocode}
\cs_set:Npn \@@_tmp:n #1
  {
    \@@_put_left_interchar_tokens:nnn {#1} { 0 }
      {
        \prop_get:NnNT \l_@@_postbreakpenalty_prop {#1} \l_@@_tmpa_tl
          {
            \tex_penalty:D \l_@@_tmpa_tl
          }
      }
  }
\seq_map_inline:Nn \g_@@_kanji_class_seq
  {
    \exp_args:Nf \@@_tmp:n { \int_eval:n {#1} }
  }
%    \end{macrocode}
%
% 和欧文間空白。
%    \begin{macrocode}
\clist_map_inline:nn
  {
    \c_xeja_kanji_ic ,
    \c_xeja_kanji_closingbrackets_ic ,
    \c_xeja_kanji_hyphens_ic ,
    \c_xeja_kanji_dividingpunctuationmarks_ic ,
    \c_xeja_kanji_middledots_ic ,
    \c_xeja_kanji_fullstops_ic ,
    \c_xeja_kanji_commas_ic ,
    \c_xeja_kanji_inseparablecharacters_ic ,
    \c_xeja_kanji_iterationmarks_ic ,
    \c_xeja_kanji_prolongedsoundmark_ic ,
    \c_xeja_kanji_smallkana_ic ,
    \c_xeja_kanji_prefixedabbreviations_ic ,
    \c_xeja_kanji_hiragana_ic ,
    \c_xeja_kanji_katakana_ic    
  }
  {
    \@@_put_right_interchar_tokens:nnn {#1} { 0 }
      {
        \skip_horizontal:n { \l_xeja_xkanji_tl }
      }
  }
%    \end{macrocode}
%
% 前の文字の字幅調整。
%    \begin{macrocode}
\@@_put_left_interchar_tokens:nnn
  { \c_xeja_kanji_closingbrackets_ic } { 0 }
  { \tex_kern:D -0.5\zw }
\@@_put_left_interchar_tokens:nnn
  { \c_xeja_kanji_middledots_ic } { 0 }
  { \tex_kern:D -0.25\zw }
\@@_put_left_interchar_tokens:nnn
  { \c_xeja_kanji_fullstops_ic } { 0 }
  { \tex_kern:D -0.5\zw }
\@@_put_left_interchar_tokens:nnn
  { \c_xeja_kanji_commas_ic } { 0 }
  { \tex_kern:D -0.5\zw }
%    \end{macrocode}
%
% フォント切り替え。
%    \begin{macrocode}
\seq_map_inline:Nn \g_@@_kanji_class_seq
  {
    \@@_put_right_interchar_tokens:nnn {#1} { 0 }
      {
        \xeja_leave_kanji_font:
      }
  }
%    \end{macrocode}
%
% \subsubsection{境界-和文間}
%
% フォント切り替え。
%    \begin{macrocode}
\seq_map_inline:Nn \g_@@_kanji_class_seq
  {
    \@@_put_left_interchar_tokens:nnn { \c_xeja_bound_ic } { #1 }
      {
        \xeja_enter_kanji_font:
      }
  }
%    \end{macrocode}
%
% \subsubsection{境界-欧文間}
%
%    \begin{macrocode}
\@@_set_interchar_tokens:nnn  { \c_xeja_bound_ic } { 0 }
  {}
%    \end{macrocode}
%
% \subsubsection{和文-境界間}
%
% フォント切り替え。
%    \begin{macrocode}
\seq_map_inline:Nn \g_@@_kanji_class_seq
  {
    \@@_put_left_interchar_tokens:nnn {#1} { \c_xeja_bound_ic }
      {
        \xeja_leave_kanji_font:
      }
  }
%    \end{macrocode}
%
% \subsubsection{欧文-境界間}
%
%    \begin{macrocode}
\@@_set_interchar_tokens:nnn  { 0 } { \c_xeja_bound_ic }
  {}
%    \end{macrocode}
%
% \subsubsection{文字クラス設定}
%
%    \begin{macrocode}
\cs_new:Npn \xeja_set_char_class:nn #1#2
  { \tex_global:D \tex_XeTeXcharclass:D \int_eval:n {#1} = \int_eval:n {#2} ~ }
\cs_new:Npn \xeja_set_char_class:nnn #1#2#3
  { \int_step_inline:nnn {#1} {#2} { \xeja_set_char_class:nn {##1} {#3} } }
%    \end{macrocode}
%
%    \begin{macrocode}
\cs_new:Npn \xeja_char_set_kanji:n #1
  { \xeja_set_char_class:nn {#1} { \c_xeja_kanji_ic } }
\cs_new:Npn \xeja_char_set_kanji:nn #1#2
  { \xeja_set_char_class:nnn {#1} {#2} { \c_xeja_kanji_ic } }
%    \end{macrocode}
%
%    \begin{macrocode}
\cs_new:Npn \xeja_char_set_alpha:n #1
  { \xeja_set_char_class:nn {#1} { \c_xeja_alpha_ic } }
\cs_new:Npn \xeja_char_set_alpha:nn #1#2
  { \xeja_set_char_class:nnn {#1} {#2} { \c_xeja_alpha_ic } }
%    \end{macrocode}
%
%    \begin{macrocode}
\xeja_char_set_kanji:nn { "0080 } { "00FF }
\xeja_char_set_kanji:nn { "0250 } { "1DFF }
\xeja_char_set_kanji:nn { "1F00 } { "D7FF }
\xeja_char_set_kanji:nn { "E000 } { "FFEF }
\xeja_char_set_kanji:nn { "10000 } { "2FFFF }
\xeja_char_set_kanji:nn { "F0000 } { "FFFFF }
\xeja_char_set_kanji:nn { "100000 } { "10FFFF }
%    \end{macrocode}
%
%    \begin{macrocode}
\xeja_char_set_alpha:nn { "0000 } { "007F } % Basic Latin
\xeja_char_set_alpha:nn { "0100 } { "017F } % Latin Extended-A
\xeja_char_set_alpha:nn { "0180 } { "024F } % Latin Extended-B
\xeja_char_set_alpha:nn { "1E00 } { "1EFF } % Latin Extended Additional
%    \end{macrocode}
%
%    \begin{macrocode}
\xeja_char_set_alpha:n { "00AA } % Feminine Ordinal Indicator
\xeja_char_set_alpha:n { "00BA } % Masculine ordinal indicator
\xeja_char_set_alpha:nn { "00C0 } { "00FF }
\xeja_char_set_kanji:n { "00D7 } % Multiplication sign
\xeja_char_set_kanji:n { "00F7 } % Division sign
%    \end{macrocode}
%
% とりあえずjlreqと同じ
%    \begin{macrocode}
\tl_map_inline:nn { (〔[{〈《「『【‘“⦅〘〖〝 }% «
  {
    \xeja_set_char_class:nn { `#1 } { \c_xeja_kanji_openingbrackets_ic }
  }
\tl_map_inline:nn { )〕]}〉》」』】’”⦆〙〗〟 }% »
  {
    \xeja_set_char_class:nn { `#1 } { \c_xeja_kanji_closingbrackets_ic }
  }
\tl_map_inline:nn { ‐〜゠– }
  {
    \xeja_set_char_class:nn { `#1 } { \c_xeja_kanji_hyphens_ic }
  }
\tl_map_inline:nn { !?‼⁇⁈⁉ }
  {
    \xeja_set_char_class:nn { `#1 }
      { \c_xeja_kanji_dividingpunctuationmarks_ic }
  }
\tl_map_inline:nn { ・:; }
  {
    \xeja_set_char_class:nn { `#1 } { \c_xeja_kanji_middledots_ic }
  }
\tl_map_inline:nn { 。. }
  {
    \xeja_set_char_class:nn { `#1 } { \c_xeja_kanji_fullstops_ic }
  }
\tl_map_inline:nn { 、, }
  {
    \xeja_set_char_class:nn { `#1 } { \c_xeja_kanji_commas_ic }
  }
\tl_map_inline:nn { —…‥〳〴〵 }
  {
    \xeja_set_char_class:nn { `#1 } { \c_xeja_kanji_inseparablecharacters_ic }
  }
\tl_map_inline:nn { ヽヾゝゞ々〻 }
  {
    \xeja_set_char_class:nn { `#1 } { \c_xeja_kanji_iterationmarks_ic }
  }
\tl_map_inline:nn { ー }
  {
    \xeja_set_char_class:nn { `#1 } { \c_xeja_kanji_prolongedsoundmark_ic }
  }
\tl_map_inline:nn { ぁぃぅぇぉァィゥェォっゃゅょゎッャュョヮヵヶゕゖㇰㇱㇲㇳㇴㇵㇶㇷㇸㇹㇺㇻㇼㇽㇾㇿ }
  {
    \xeja_set_char_class:nn { `#1 } { \c_xeja_kanji_smallkana_ic }
  }
\tl_map_inline:nn { ¥$£#€№ }
  {
    \xeja_set_char_class:nn { `#1 } { \c_xeja_kanji_prefixedabbreviations_ic }
  }
\tl_map_inline:nn { あいうえおかがきぎくぐけげこごさざしじすずせぜそぞただちぢつづてでとどなにぬねのはばぱひびぴふぶぷへべぺほぼぽまみむめもやゆよらりるれろわゐゑをんゔ }
  {
    \xeja_set_char_class:nn { `#1 } { \c_xeja_kanji_hiragana_ic }
  }
\tl_map_inline:nn { アイウエオカガキギクグケゲコゴサザシジスズセゼソゾタダチヂツヅテデトドナニヌネノハバパヒビピフブプヘベペホボポマミムメモヤユヨラリルレロワヰヱヲンヴヷヸヹヺ }
  {
    \xeja_set_char_class:nn { `#1 } { \c_xeja_kanji_katakana_ic }
  }
%    \end{macrocode}
%
% \subsection{ボックス}
%
%    \begin{macrocode}
\cs_set_eq:NN \@@_special:n \tex_special:D
\cs_new:Npn \xeja_graphics_save:
  { \@@_special:n { x:gsave } }
\cs_new:Npn \xeja_graphics_restore:
  { \@@_special:n { x:grestore } }
\cs_new:Npn \xeja_graphics_rotate:n #1
  { \@@_special:n { x:rotate~ #1 } }
%    \end{macrocode}
%
%    \begin{macrocode}
\box_new:N \l_@@_rotate_box
\dim_new:N \l_@@_rotate_box_ht_dim
\dim_new:N \l_@@_rotate_box_dp_dim
\dim_new:N \l_@@_rotate_box_wd_dim
%    \end{macrocode}
%
% \begin{macro}{\@@_rotate_box_tate_in_yoko:N}
% ボックスを時計回りに90度回転する。^^A
% 回転後のボックス下端がベースラインになる。
%    \begin{macrocode}
\cs_new:Npn \@@_rotate_box_tate_in_yoko:N #1
  {
%    \end{macrocode}
% 元のボックスの寸法を取得する。
%    \begin{macrocode}
    \dim_set:Nn \l_@@_rotate_box_ht_dim { \box_ht:N #1 }
    \dim_set:Nn \l_@@_rotate_box_dp_dim { \box_dp:N #1 }
    \dim_set:Nn \l_@@_rotate_box_wd_dim { \box_wd:N #1 }
%    \end{macrocode}
% 元のボックスの右端が回転後にベースラインに来るように位置調整する。
%    \begin{macrocode}
    \hbox_set:Nn \l_@@_rotate_box
      {
        \tex_kern:D -\l_@@_rotate_box_wd_dim
        \box_use:N #1
      }
%    \end{macrocode}
% ボックスを時計回りに90度回転する。
%    \begin{macrocode}
    \hbox_set:Nn \l_@@_rotate_box
      {
        \xeja_graphics_save:
        \xeja_graphics_rotate:n { -90 }
        \box_use:N \l_@@_rotate_box
        \xeja_graphics_restore:
      }
%    \end{macrocode}
% 元のボックスの下端が左端になるように位置調整する。
%    \begin{macrocode}
    \hbox_set:Nn \l_@@_rotate_box
      {
        \tex_kern:D \l_@@_rotate_box_dp_dim
        \box_use:N \l_@@_rotate_box
      }
%    \end{macrocode}
% ボックス寸法を調整する。
%    \begin{macrocode}
    \box_set_ht:Nn \l_@@_rotate_box
      { \l_@@_rotate_box_wd_dim }
    \box_set_dp:Nn \l_@@_rotate_box { 0pt }
    \box_set_wd:Nn \l_@@_rotate_box
      { \l_@@_rotate_box_ht_dim + \l_@@_rotate_box_dp_dim }
%    \end{macrocode}
%
%    \begin{macrocode}
    \box_set_eq_drop:NN #1 \l_@@_rotate_box
  }
%    \end{macrocode}
% \end{macro}
%
% \begin{macro}{\@@_rotate_box_yoko_in_tate:N}
% ボックスを反時計回りに90度回転する。^^A
% 回転後のボックス中央がベースラインになる。
%    \begin{macrocode}
\cs_new:Npn \@@_rotate_box_yoko_in_tate:N #1
  {
%    \end{macrocode}
% 元のボックスの寸法を取得する。
%    \begin{macrocode}
    \dim_set:Nn \l_@@_rotate_box_ht_dim { \box_ht:N #1 }
    \dim_set:Nn \l_@@_rotate_box_dp_dim { \box_dp:N #1 }
    \dim_set:Nn \l_@@_rotate_box_wd_dim { \box_wd:N #1 }
%    \end{macrocode}
% 元のボックスの中央が回転後にベースラインに来るように位置調整する。
%    \begin{macrocode}
    \hbox_set:Nn \l_@@_rotate_box
      {
        \tex_kern:D -0.5\l_@@_rotate_box_wd_dim
        \box_use:N #1
      }
%    \end{macrocode}
% ボックスを反時計回りに90度回転する。
%    \begin{macrocode}
    \hbox_set:Nn \l_@@_rotate_box
      {
        \xeja_graphics_save:
        \xeja_graphics_rotate:n { 90 }
        \box_use:N \l_@@_rotate_box
        \xeja_graphics_restore:
      }
%    \end{macrocode}
% 元のボックスの上端が左端になるように位置調整する。
%    \begin{macrocode}
    \hbox_set:Nn \l_@@_rotate_box
      {
        \tex_kern:D \l_@@_rotate_box_ht_dim
        \box_use:N \l_@@_rotate_box
      }
%    \end{macrocode}
% ボックス寸法を調整する。
%    \begin{macrocode}
    \box_set_ht:Nn \l_@@_rotate_box
      { 0.5\l_@@_rotate_box_wd_dim }
    \box_set_dp:Nn \l_@@_rotate_box
      { 0.5\l_@@_rotate_box_wd_dim }
    \box_set_wd:Nn \l_@@_rotate_box
      { \l_@@_rotate_box_ht_dim + \l_@@_rotate_box_dp_dim }
%    \end{macrocode}
%
%    \begin{macrocode}
    \box_set_eq_drop:NN #1 \l_@@_rotate_box
  }
%    \end{macrocode}
% \end{macro}
%
%    \begin{macrocode}
\box_new:N \l__xeja_tate_yoko_box
%    \end{macrocode}
%
%    \begin{macrocode}
\cs_new:Npn \@@_yoko_in_tate_hbox:n #1
  {
    \hbox_set:Nn \l__xeja_tate_yoko_box
      { \bool_set_false:N \l_xeja_tate_text_bool #1 }
    \@@_rotate_box_yoko_in_tate:N \l__xeja_tate_yoko_box
    \box_use_drop:N \l__xeja_tate_yoko_box
  }
%    \end{macrocode}
%
%    \begin{macrocode}
\cs_new:Npn \@@_tate_in_yoko_hbox:n #1
  {
    \hbox_set:Nn \l__xeja_tate_yoko_box
      { \bool_set_true:N \l_xeja_tate_text_bool #1 }
    \@@_rotate_box_tate_in_yoko:N \l__xeja_tate_yoko_box
    \box_use_drop:N \l__xeja_tate_yoko_box
  }
%    \end{macrocode}
%
%    \begin{macrocode}
\cs_new:Npn \xeja_yoko_hbox:n #1
  {
    \bool_if:NTF \l_xeja_tate_text_bool
      { \@@_yoko_in_tate_hbox:n {#1} }
      { \hbox:n {#1} }
  }
%    \end{macrocode}
%
%    \begin{macrocode}
\cs_new:Npn \xeja_tate_hbox:n #1
  {
    \bool_if:NTF \l_xeja_tate_text_bool
      { \hbox:n {#1} }
      { \@@_tate_in_yoko_hbox:n {#1} }
  }
%    \end{macrocode}
%
% \subsection{ページ出力}
%
%    \begin{macrocode}
\cs_set_eq:NN \@@_original_output_page: \@outputpage
\cs_set:Npn \@outputpage
  {
    \bool_if:NT \g_xeja_tate_document_bool
      { \@@_rotate_box_tate_in_yoko:N \@outputbox }
    \bool_set_false:N \l_xeja_tate_text_bool
    \@@_original_output_page:
    \bool_set_eq:NN \l_xeja_tate_text_bool \g_xeja_tate_document_bool
  }
%    \end{macrocode}
%
%    \begin{macrocode}
\hook_gput_code:nnn { begindocument } { xelatexja / tate }
  {
    \bool_if:NT \g_xeja_tate_document_bool
      { \bool_set_true:N \l_xeja_tate_text_bool }
  }
%    \end{macrocode}
%
%    \begin{macrocode}
%</package>
%    \end{macrocode}
%
% \end{implementation}
%
% \PrintIndex
