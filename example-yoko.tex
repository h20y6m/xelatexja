%#!xelatex
\documentclass[a4paper,twocolumn,tombow]{xltjarticle}

\usepackage{xltjext}

\usepackage{array,amsmath}
\usepackage{pxrubrica}
\usepackage{bxjalipsum}
\usepackage{bxtexlogo}
\bxtexlogoimport{*}

\title{\XeLaTeX-ja横組みサンプル}
\author{森見幸正 (h20y6m)}
\西暦

\begin{document}
\maketitle

\section{数式}

二次方程式$ax^2+bx+c=0$の解は、
\[ x = \frac{-b\pm\sqrt{b^2+4ac}}{2a} \]
で与えられる。

\section{ルビ}

\ruby{吾輩}{わが|はい}は\ruby{猫}{ねこ}である。\ruby{名前}{な|まえ}はまだ\ruby{無}{ない}い。

どこで\ruby{生}{うま}れたかとんと\ruby{見当}{けん|とう}がつかぬ。

\section{縦横}

\begin{center}\scriptsize
\begin{tabular}{c|c|c|c|c|c|}
\texttt{\texttt{\textbackslash pbox}} &
                  &
\texttt{[3zw][l]} &
\texttt{[3zw][c]} &
\texttt{[3zw][r]} \\
\hline
             &
①\pbox{②③}④ &
①\pbox[3\zw][l]{②③}④ &
①\pbox[3\zw][c]{②③}④ &
①\pbox[3\zw][r]{②③}④ \\
\hline
\texttt{<y>} &
①\pbox<y>{②③}④ &
①\pbox<y>[3\zw][l]{②③}④ &
①\pbox<y>[3\zw][c]{②③}④ &
①\pbox<y>[3\zw][r]{②③}④ \\
\hline
\texttt{<t>} &
①\pbox<t>{②③}④ &
①\pbox<t>[3\zw][l]{②③}④ &
①\pbox<t>[3\zw][c]{②③}④ &
①\pbox<t>[3\zw][r]{②③}④ \\
\hline
\texttt{<z>} &
①\pbox<z>{②③}④ &
①\pbox<z>[3\zw][l]{②③}④ &
①\pbox<z>[3\zw][c]{②③}④ &
①\pbox<z>[3\zw][r]{②③}④ \\
\hline
\end{tabular}
\end{center}

\section{結合文字}

\subsection{異体字セレクタ}

渡邉邉󠄀邉󠄁邉󠄂邉󠄃邉󠄄邉󠄅邉󠄆邉󠄇邉󠄈邉󠄉邉󠄊邉󠄋邉󠄌邉󠄍邉󠄎さんと
渡邊邊󠄀邊󠄁邊󠄂邊󠄃邊󠄄邊󠄅邊󠄆邊󠄇さん。

\subsection{濁点・半濁点}

か゚き゚く゚け゚こ゚、カ゚キ゚ク゚ケ゚コ゚、セ゚ツ゚ト゚ㇷ゚。

\newpage
\section{BXjalipsumダミーテキスト}

\subsection{いろは歌}
\jalipsum{iroha}

\subsection{寿限無}
\jalipsum{jugemu}

\subsection{吾輩は猫である}
\jalipsum{wagahai}

\subsection{日本国憲法前文}
\jalipsum{preamble}

\subsection{初恋}
\jalipsum{hatsukoi}

\subsection{草枕}
\jalipsum{kusamakura}

\end{document}
