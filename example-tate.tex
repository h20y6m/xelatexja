%#!xelatex
\documentclass[a4paper,twocolumn,tombow]{xltjtarticle}

\usepackage{amsmath}
\usepackage{pxrubrica}
\usepackage{bxjalipsum}
\usepackage{bxtexlogo}
\bxtexlogoimport{*}

\title{\XeLaTeX-ja縦組みサンプル}
\author{森見幸正 (h20y6m)}
\和暦

\begin{document}
\maketitle

\section{数式}

二次方程式$ax^2+bx+c=0$の解は、
\[ x = \frac{-b\pm\sqrt{b^2+4ac}}{2a} \]
で与えられる。

\section{ルビ}

\ruby{吾輩}{わが|はい}は\ruby{猫}{ねこ}である。\ruby{名前}{な|まえ}はまだ\ruby{無}{ない}い。

どこで\ruby{生}{うま}れたかとんと\ruby{見当}{けん|とう}がつかぬ。

\section{いろは歌}
\jalipsum{iroha}

\section{寿限無}
\jalipsum{jugemu}

\section{吾輩は猫である}
\jalipsum{wagahai}

\section{日本国憲法前文}
\jalipsum{preamble}

\section{初恋}
\jalipsum{hatsukoi}

\section{草枕}
\jalipsum{kusamakura}

\end{document}
