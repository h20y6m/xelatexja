%#!xelatex
\documentclass[a4paper,twocolumn]{article}
\usepackage[tate]{xelatexja}
\usepackage{xltjext}

\usepackage{amsmath}
\usepackage{pxrubrica}
\usepackage{bxjalipsum}
\usepackage{bxtexlogo}
\bxtexlogoimport{*}


\usepackage{indentfirst}

\RenewDocumentCommand{\normalsize}{}{\fontsize{10pt}{15pt}}
\normalsize
\setlength{\parindent}{1\zw}
\setlength{\topskip}{1\zw}
\setlength{\parskip}{0pt}
\setlength{\columnsep}{2\zw}

\ExplSyntaxOn
\int_set:Nn \l_tmpa_int
  { \int_div_truncate:nn { \textheight - \columnsep } { 2 * \zw } * 2 }
\int_set:Nn \l_tmpb_int
  { \int_div_truncate:nn { \textwidth - \topskip } { \baselineskip } }
\dim_set:Nn \textwidth
  { 1\zw * \l_tmpa_int + \columnsep }
\dim_set:Nn \textheight
  { \baselineskip * \l_tmpb_int + \topskip }
\ExplSyntaxOff

\renewcommand{\thesection}{\rensuji{\arabic{section}}}

\xejaTombowSetup{tombow=true}
\setlength{\pdfpagewidth}{\paperwidth}
\setlength{\pdfpageheight}{\paperheight}
\addtolength{\pdfpagewidth}{2in}
\addtolength{\pdfpageheight}{2in}


\title{\XeLaTeX-ja縦組みサンプル}
\author{森見幸正 (h20y6m)}
\date{令和三年九月五日}

\begin{document}
\maketitle

\section{数式}

二次方程式$ax^2+bx+c=0$の解は、
\[ x = \frac{-b\pm\sqrt{b^2+4ac}}{2a} \]
で与えられる。

\section{ルビ}

\ruby{吾輩}{わが|はい}は\ruby{猫}{ねこ}である。\ruby{名前}{な|まえ}はまだ\ruby{無}{ない}い。

どこで\ruby{生}{うま}れたかとんと\ruby{見当}{けん|とう}がつかぬ。

\section{いろは歌}
\jalipsum{iroha}

\section{寿限無}
\jalipsum{jugemu}

\section{吾輩は猫である}
\jalipsum{wagahai}

\section{日本国憲法前文}
\jalipsum{preamble}

\section{初恋}
\jalipsum{hatsukoi}

\section{草枕}
\jalipsum{kusamakura}

\end{document}
